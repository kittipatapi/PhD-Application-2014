% LaTeX file for resume 
% This file uses the resume document class (res.cls)

\documentclass{res} 
%\usepackage{helvetica} % uses helvetica postscript font (download helvetica.sty)
%\usepackage{newcent}   % uses new century schoolbook postscript font 
\newsectionwidth{0pt}  % So the text is not indented under section headings
\usepackage{fancyhdr}  % use this package to get a 2 line header
\usepackage{hyperref}  % use this package to provide url links
%\usepackage{mdwlist}	% use this package for customizing lists
\renewcommand{\headrulewidth}{0pt} % suppress line drawn by default by fancyhdr
\setlength{\headheight}{24pt} % allow room for 2-line header
\setlength{\headsep}{24pt}  % space between header and text
\setlength{\headheight}{24pt} % allow room for 2-line header
\pagestyle{fancy}     % set pagestyle for document
\rhead{ {\it K. Apicharttrisorn's Resume}\hspace{0.2in}{\it p. \thepage} } % put text in header (right side)
\cfoot{}                                     % the foot is empty
\topmargin=-0.5in % start text higher on the page

\begin{document}
\thispagestyle{empty} % this page has no header  
\name{KITTIPAT APICHARTTRISORN\\[12pt]}% the \\[12pt] adds a blank line after name

\address{{\bf Office Address} \\ Air Traffic Data Systems Engineering Department \\
  Aeronautical Radio of Thailand \\   Sathon, Bangkok, Thailand 10120 \\ (+66) 2285-9177}    
                                      
\address{{\bf Permanent Address} \\ 7/639 Vibhavadee-Rangsit Rd. \\ Chatuchak, Bangkok 10900\\
  (+66) 2537-0097}



\begin{resume}
 
\section{\centerline{OBJECTIVE}}
\vspace{8pt} % provide vertical space between section title and contents
A Ph.D. student position in Computer Science with research interest in computer networks, resource allocation, software-defined networking, sensor networks, and internet of things. 
 
\vspace{0.2in}
\section{\centerline{EDUCATION}} 
\vspace{8pt} 
{\sl Master of Science}, Computer Science \\
Chulalongkorn University, Bangkok, Thailand \hspace{0.2in}  GPA 3.75 / 4.00 \hfill November 2010 \\
THESIS - Distributed Time Synchronization in Wireless Sensor Networks \\
ADVISOR - Asst. Prof. Dr. Chalermek Intanagonwiwat
 
{\sl Bachelor of Engineering}, Electrical Engineering \\ % \sl will be bold italic in
					 % New Century Schoolbook (or
					 % any postscript font) and
					 % just slanted in Computer
					 % Modern (default) font
Kasetsart University, Bangkok, Thailand  \hspace{0.2in}  GPA 2.49 / 4.00    \hfill    October 2004
  
\vspace{0.2in} 
\section{\centerline{EMPLOYMENT}} 
\vspace{8pt}
{\sl Senior Systems Engineer} \hfill        January 2007 - Present \\
Air Traffic Data Systems Engineering Department \\
  Aeronautical Radio of Thailand, Bangkok, Thailand       
  
\begin{itemize} \itemsep -2pt % reduce space between items
   \item Developed  new selection criteria for applicant screening and 
    selection  at  GPD  Tucson.  Conducted  job  analysis,  wrote 
    criteria,  identified  skill  codes  for  applicant tracking, 
    established rater reliability. 
   \item Interviewed   applicants   for   positions    in    Assembly, 
    Warehousing, and Direct Customer Response. 
  \item Assistant   Co-op   Coordinator.   Initiated  and  maintained 
    computer tracking for  Co-op  program.  Organized  all  co-op 
    seminars and activities, co-op directory. 
   \item Representative  on  GPD Compensation Task Force. Prepared job 
    descriptions, assigned corporate position code, and submitted 
    for division approval. 
 \end{itemize}
 
{\sl Network Engineer} \hfill                  March 2005 - September 2006 \\
1tonet Co., Ltd., Bangkok, Thailand
  \begin{itemize} \itemsep -2pt
  \item Served  as  consultant  for new management team in techniques 
    for managing change. 
 \item  Developed and administered organizational climate survey. 
 \item  Facilitated management-employee feedback sessions. 
 \end{itemize}
                    
\vspace{0.2in} 
\section{\centerline{PUBLICATIONS}} 
\vspace{18pt}
\begin{itemize}
	\item \textbf{``A Stable and Equitable Desynchronization Algorithm for Multi-Hop Wireless Sensor Networks''}
		\begin{description}\itemsep -2pt 
		\item[Authors]Supasate Choochaisri, Kittipat Apicharttrisorn and Chalermek Intanagonwiwat
		\item[Publication Name]ACM Transactions on Sensor Networks (TOSN)
		\item[Publication Date]\textit{Under submission}
		\item[Abstract]{\sl \small  N/A }
		\end{description}
	\vspace{2pt}
	\item \textbf{``A Moving Object Tracking Algorithm Using Support Vector Machines in Binary Sensor Networks''}
		\begin{description}\itemsep -2pt 
		\item[Authors]Dusadee Apicharttrisorn, Kittipat Apicharttrisorn and Teerasit Kasetkasem
		\item[Publication Name]The 13th International Symposium on Communications and Information Technologies
		\item[Publication Date]\textit{To appear}
		\item[Abstract]{\sl \small Wireless sensor technologies have enabled us to deploy such small sensors to monitor an area of interest. Object tracking is one of the most attractive applications to be implemented with wireless sensor networks (WSNs). However, many solutions are struggled with energy-draining global positioning system (GPS), poorly-performed trilateration for indoor usage, and impractical, complex algorithms to be implemented in sensor nodes. This paper proposes a moving object tracking algorithm using support vector machines (MOT-SVM). The MOT-SVM takes advantage of light-weighted directional binary sensor networks, and state-of-the-art signal processing algorithms, namely the support vector machines and particle filters. We compare our proposed algorithm with the Aslam's work [1] through the simulation. We examine our algorithms for various movement scenarios such as the linear, random and the 8-model trajectories, and the scenarios in which observing sensors make observation errors.}
		\end{description}
   \vspace{2pt}
	\item \textbf{``Desynchronization with an artificial force field for wireless networks''}
		\begin{description}\itemsep -2pt 
		\item[Authors]Supasate Choochaisri, Kittipat Apicharttrisorn, Kittiporn Korprasertthaworn, Pongpakdi Taechalertpaisarn and  Chalermek Intanagonwiwat
		\item[Publication Name]SIGCOMM Computer Communication Review
		\item[Publication Date]March 2012 
		\item[Abstract]{\sl \small Desynchronization is useful for scheduling nodes to perform tasks at different time. This property is desirable for resource sharing, TDMA scheduling, and collision avoiding. Inspired by robotic circular formation, we propose DWARF (Desynchronization With an ARtificial Force field), a novel technique for desynchronization in wireless networks. Each neighboring node has artificial forces to repel other nodes to perform tasks at different time phases. Nodes with closer time phases have stronger forces to repel each other in the time domain. Each node adjusts its time phase proportionally to its received forces. Once the received forces are balanced, nodes are desynchronized. We evaluate our implementation of DWARF on TOSSIM, a simulator for wireless sensor networks. The simulation results indicate that DWARF incurs significantly lower desynchronization error and scales much better than existing approaches.}
		\end{description}
 	\vspace{2pt}
	\item \textbf{``Energy-Efficient Gradient Time Synchronization for Wireless Sensor Networks''}
		\begin{description}\itemsep -2pt 
		\item[Authors]Kittipat Apicharttrisorn, Supasate Choochaisri and Chalermek Intanagonwiwat
		\item[Publication Name]2010 Second International Conference on Computational Intelligence, Communication Systems and Networks (CICSyN)
		\item[Publication Date]July 2010
		\item[Abstract]{\sl \small Wireless sensor network (WSN) applications usually demand a time-synchronization protocol for node coordination and data interpretation. In this paper, we propose an Energy-Efficient Gradient Time Synchronization Protocol (EGTSP) for Wireless Sensor Networks. In contrast to FTSP, a state-of-the-art synchronization protocol for WSNs, EGTSP is a completely localized algorithm that achieves a global time consensus and gradient time property using effective drift compensation and incremental averaging estimation. In contrast with GTSP, a gradient-based fixed-rated time synchronization protocol, our protocol provides adaptive beaconing for applications to optimize energy savings by selecting appropriate message-broadcast periods. The protocol is implemented and evaluated on multi-hop networks that consist of Telosb motes running TinyOS. The experimental results indicate that our protocol achieves a network-wide global notion of time, attains small synchronization errors, and utilizes energy efficiently.}
		\end{description}
  \end{itemize}

\vspace{0.2in} 
\section{\centerline{ACADEMIC PROJECTS}} 
\vspace{18pt}
	\begin{itemize} \itemsep -2pt
		\item Project Name: Time Synchronization for Wireless Sensor Networks
			\begin{description} \itemsep -2pt 
			\item[Objective] MS Thesis's Research Project 
			\item[Description]
			\item[Period] January 2008 - October 2010
			\item[Roles and Responsibility]				
			\item[Tools and Environments] TinyOS, Ubuntu, Gnuplot
			\end{description}		
		\item Project Name: Desynchronization as Distributed Resource Allocations and TDMA
			\begin{description} \itemsep -2pt
				\item[Objective] Research Project 
				\item[Description]
				\item[Period] March 2010 - Present
				\item[Roles and Responsibility]				
				\item[Tools and Environments]
			\end{description}
		\item Project Name: Moving Object Tracking using Support Vector Machine in Binary Sensor Networks
			\begin{description} \itemsep -2pt
				\item[Objective] Research Project
				\item[Description] 
				\item[Period] March 2013 - Present
				\item[Roles and Responsibility]				
				\item[Tools and Environments]
			\end{description}		
		\item Project Name: Distributed Online Ticket Reservation with Display on Google Maps
			\begin{description} \itemsep -2pt
				\item[Objective] Graduate Course Project (Distributed Systems)
				\item[Description]
				\item[Period] June 2008 - October 2008
				\item[Roles and Responsibility]				
				\item[Tools and Environments]
			\end{description}							
		\item Project Name: Undergrad Admission Systems: Information Systems Architecture
			\begin{description} \itemsep -2pt
				\item[Objective] Graduate Course Project (Information Systems Architecture)
				\item[Description]
				\item[Period] June 2007 - October 2007
				\item[Roles and Responsibility]				
				\item[Tools and Environments] MS Words, MS Visio
			\end{description}
		\item Project Name: Adaptive Multi-Rate - Wideband (AMR-WB) speech codec Testing
			\begin{description} \itemsep -2pt
				\item[Objective] Undergraduate Senior Project (Electrical Engineering Project)
				\item[Description]
				\item[Period] June 2003 - Mar 2004
				\item[Roles and Responsibility]				
				\item[Tools and Environments] MS Visual C
			\end{description}	
	\end{itemize}
		
\vspace{0.2in} 
\section{\centerline{PROFESSIONAL PROJECTS}} 
\vspace{18pt}
\begin{itemize} \itemsep -2pt
		\item Project Name: Aeronautical Message Switching Systems (AMSS)
		\begin{description} \itemsep -2pt 
			\item[Description] AMSS is a core aeronautical data system that switch, store and manipulate data that are sent and received by aeronautical units worldwide so that flights are operated and managed properly and according to ICAO's regulations.
			\item[Period] January 2008 - January 2010
			\item[Roles and Responsibilities]
			\item[Tools and Environments] Redhat Enterprise, Windows Servers, Oracle DBMS, 
		\end{description}
		\item Project Name: Aeronautical Message Handling Systems (AMHS) and X.400
		\begin{description} \itemsep -2pt 
			\item[Description] AMSS is a core aeronautical data system that switch, store and manipulate data that are sent and received by aeronautical units worldwide so that flights are operated and managed properly and according to ICAO's regulations.
			\item[Period] January 2008 - January 2010
			\item[Roles and Responsibilities]
			\item[Tools and Environments] Redhat Enterprise, Windows Servers, Oracle DBMS,	
		\end{description}
		\item Project Name: Aeronautical Message Handling Systems (AMHS) and X.400
		\begin{description} \itemsep -2pt 
			\item[Description] AMSS is a core aeronautical data system that switch, store and manipulate data that are sent and received by aeronautical units worldwide so that flights are operated and managed properly and according to ICAO's regulations.
			\item[Period] January 2008 - January 2010
			\item[Roles and Responsibilities]
			\item[Tools and Environments] Redhat Enterprise, Windows Servers, Oracle DBMS,	
		\end{description}
		\item Project Name: Aeronautical Message Handling Systems (AMHS) and X.400
		\begin{description} \itemsep -2pt 
			\item[Description] AMSS is a core aeronautical data system that switch, store and manipulate data that are sent and received by aeronautical units worldwide so that flights are operated and managed properly and according to ICAO's regulations.
			\item[Period] January 2008 - January 2010
			\item[Roles and Responsibilities]
			\item[Tools and Environments] Redhat Enterprise, Windows Servers, Oracle DBMS,	
		\end{description}
		\item Project Name: Aeronautical Message Handling Systems (AMHS) and X.400
		\begin{description} \itemsep -2pt 
			\item[Description] AMSS is a core aeronautical data system that switch, store and manipulate data that are sent and received by aeronautical units worldwide so that flights are operated and managed properly and according to ICAO's regulations.
			\item[Period] January 2008 - January 2010
			\item[Roles and Responsibilities]
			\item[Tools and Environments] Redhat Enterprise, Windows Servers, Oracle DBMS,	
		\end{description}	
\end{itemize}
	
\vspace{0.2in} 
\section{\centerline{GRANTS}} 
\vspace{15pt}
 \begin{itemize} \itemsep -2pt 
	\item Grant Name: International Conference Attendance Support Grants for Graduate Students
		\indent \begin{description} \itemsep -2pt \topsep 0pt
			\item[Period] July 2010
			\item[Purpose]
			\item[Amount] Approximately 1200 USD 
			\item[Granted by] Graduate School, Chulalongkorn University Bangkok, Thailand
		\end{description}
	\item Grant Name: AINTEC (ASIAN INTERNET ENGINEERING CONFERENCE) Conference Attendance Grants
		\begin{description} \itemsep -2pt 
			\item[Period] November 2010
			\item[Purpose]
			\item[Amount] Attendance Fee (Unknown) 
			\item[Granted by]  Thailand Research Education Network Association (ThaiREN), Bangkok, Thailand
		\end{description}
	\item Grant Name: AINTEC (ASIAN INTERNET ENGINEERING CONFERENCE) Conference Attendance Grants
		\begin{description} \itemsep -2pt 
			\item[Period] November 2008
			\item[Purpose]
			\item[Amount] Attendance Fee (Unknown) 
			\item[Granted by]  Thailand Research Education Network Association (ThaiREN), Bangkok, Thailand
		\end{description}
\end{itemize} 
 
\vspace{0.2in} 
\section{\centerline{ACADEMIC ACTIVITIES}} 
\vspace{15pt}
		\begin{itemize}
		\item External Reviewer: IEEE International Conference on Computer Communications (INFOCOM 2011) \\
		       \indent Delegated by Asst. Prof. Dr. Chalermek Intanagonwiwat
		\item External Reviewer: IEEE International Conference on Computer Communications (INFOCOM 2012) \\
		 		Delegated by Asst. Prof. Dr. Chalermek Intanagonwiwat
		\end{itemize}

\vspace{0.2in} 
 \section{\centerline{ CERTIFICATES }}
 \vspace{15pt} 
\begin{itemize}
 \item Network Design and Implementation I
 \item Certified Thaicom Users
\end{itemize}

\vspace{0.2in} 
\section{\centerline{ SKILLS }}
\vspace{8pt} 
\normalsize{\section{Programming Languages}}
                 \begin{itemize}
                 \item C, C++, NesC, TinyOS, Matlab, Java, Python, SQL
                 \end{itemize}
\normalsize{\section{Computer Software}} 
                 \begin{itemize}
                  \item Ubuntu, UNIX, Gnuplot, Latex. 
                  \end{itemize}
\normalsize{\section{Language Proficiency}}
	           \begin{itemize} 
                   \item English: TOEFL 104 iBT  (Test Date: 25 Aug 2013) \\ Reading: 28, Listening: 26, Speaking: 22, Writing: 28
                  \item Thai: Native 
                   \end{itemize}

\section{\centerline{VOLUNTEER SERVICES}} 
\vspace{15pt}
\begin{itemize}
\item Event Name: CANSO Global ATM Summit and 15th Annual General Meeting (AGM)
\begin{description} \itemsep -2pt 
\item[Period] 11 June 2011 - 14 June 2011
\item[Description]:
\item[Contributions]:
\item[Benefits]:
\end{description}

\end{itemize}

\vspace{0.2in} 
\section{\centerline{REFERENCES}} 
\vspace{15pt}
		\begin{itemize}
		\item Air Chief Marshal Somchai Thean-anant 
		\begin{description}\itemsep -2pt 
			\item[Position]Former President of Aeronautical Radio of Thailand
			\item[Address]
			\item[Email]
			\item[Tel.]
		\end{description}
		\item Dr. Chalermek Intanagonwiwat
		\begin{description}\itemsep -2pt 
			\item[Position]
			\item[Address]
			\item[Email]email: intanago@yahoo.com
			\item[Tel.]
		\end{description}				
		\item Mr. Pongnarin Anantasirijinda
			\begin{description}\itemsep -2pt 
				\item[Position]Director of Air Traffic Data Systems Engineering Department
				\item[Address]Aeronautical Radio of Thailand, Bangkok, Thailand, 10120
				\item[Email]add@aerothai.co.th
				\item[Tel.](+66)02-285-9101 
			\end{description}		        
		\item Asst. Prof. Dr. Teerasit Kasetkasem
			\begin{description}\itemsep -2pt 
				\item[Position]Assistant Professor
				\item[Address]Kasetsart University, Bangkok, Thailand, 10900
				\item[Email]fengtsk@ku.ac.th
				\item[Tel.](+66)02-942-8555 ext 1536
			\end{description}
		 \item Dr. Supasate Choochaisri
		 	\begin{description}\itemsep -2pt 
		 		\item[Position]Chief Technology Officer
		 		\item[Address]Software Park Bld. Nonthaburi, Thailand 11120
		 		\item[Email]
		 		\item[Tel.](+66)02-584-6064
		 	\end{description}
		 \item 
		\end{itemize}

\vspace{0.2in}
\section{\centerline{INTERESTS AND HOBBIES}} 
\vspace{-5pt} 
\begin{center}
Jazz and blues guitar, photography,  bodybuilding/aerobic  exercise,  cooking, swimming
\end{center} 
 
\end{resume}

\end{document}













