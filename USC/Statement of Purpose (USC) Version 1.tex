\documentclass[a4paper,10pt]{report}
\hyphenation{mathe-matics under-graduate know-ledge}
\usepackage{fullpage}
%\usepackage{helvetica}
\newcommand{\university}{University of Southern California}
\newcommand{\department}{Department of Computer Science}
\newcommand{\uniabbre}{USC}
\newcommand{\labfirst}{Networked Systems Laboratory}
\newcommand{\labfirstabbre}{NSL}
\begin{document}
\begin{center}
\textbf{\large Statement of Purpose}
\end{center}

\vspace{0.4cm}
This statement of purpose is intended for use with my application to the Philosophy of Doctor graduate program at the \department, \university. This document starts by portraying my education background, both the Bachelor and Master's degrees. Then, it briefly states my professional experience and explains my research experience during my Master's degree study. Then my interest in \uniabbre's teaching and research is elaborated and finally my future plans after graduation are described. After finishing reading this statement of purpose, the committee will learn why I am qualified to be an excellent student of the program, what motivates me to pursue the doctoral degree at \uniabbre \space and why it is so important for my future profession that I earn this degree.

\vspace{0.2cm}
During my undergraduate study, in addition to a number of Electrical Engineering subjects, I studied a wide range of mathematical subjects including four Calculus courses, a course on Probability, and another on Linear Algebra and Complex Numbers, all of which are basic principles of Computer Science. Moreover, I passed two courses on computer programming, data structures and algorithms, which are the knowledge crucial to a successful computer scientist. However, during the first three years of the study, although I enjoyed learning the subjects, I was so shiftless and unmotivated that I did not pay much attention to my academic records. Not until the beginning of the forth year did I decide to boost my GPA as I was conscious that after that year I had to apply for a job and the currently low GPA would preclude me from competing with other students. This consciousness encouraged me to attend classes more frequently, pay more attention to the study materials, and better prepare for the examinations. As a result, my semester GPAs of the forth year were able to stay in a good standing until I graduated. Unfortunately, the total GPA was unable to increase much and remained at 2.49/4.00. 

\vspace{0.2cm}
After graduation with the Bechelor of Electrical Engineering degree, I had to apply for a job to earn a living and support my family. I had worked for three companies until I settled my career at Aeronautical Radio of Thailand or Aerothai, a state enterprise under the Ministry of Transport, Thailand. One of Aerothai's principal missions is to provide air navigation services or air traffic control within Thailand's airspace. Specifically, the department of air traffic data systems engineering is responsible for the provision of data systems that support air traffic controllers' operations efficiently and continuously. At the department, my colleagues and I design, configure, and implement those systems by taking advantage of enterprise-graded information technology products, mostly of the USA, such as HP and Dell servers, Oracle and Microsoft databases, Cisco network equipment, and VMWare's virtualization technology, etc. Therefore, I have witnessed how these innovative products help transform air traffic data systems into more reliable and efficient systems. This hand-on, seven-year experience allows me to learn practical aspects of enterprise information systems with our safety-critical applications, and makes me interested in computer science, a core foundation of computer-related products and services. Moreover, the exposure to these technologies encourages me to plan to further my study in the US.

\vspace{0.2cm}
Not long after I started work for Aerothai did I decide to continue my education to the Master's degree in Computer Science at the Department of Computer Engineering, Chulalongkorn University. I had the following three main reasons. First, as a computer systems engineer, studying computer science would give me a professional advantage in terms of the received degree and knowledge. Second, this thesis-based curriculum would allow me to gain research experience in computer science, which would be crucial to my PhD study in the future. Third, in this program, I would have a chance to study a wide range of computer science subjects from Theory of Computation and Computer Algorithms to Computer Networks and Distributed Systems. During the study, I worked hard on studying materials, undertaking term projects and making progress toward my thesis work. As a result, I was able to earn a very good GPA of 3.75/4.00 in the Master's degree with the complete thesis titled ``Distributed Time Sychronization for Wireless Sensor Networks''.

\vspace{0.2cm}
My decision to pursue the Master's degree was correct because I gained a lot of valuable research experience there. At the department, I was a member of the Ubiquitous Network laboratory under the supervision of Associate Professor Dr. Chalermek Intanagonwiwat, who was also my advisor. At this lab, I learned at least three priceless lessons of research experience. First, I learned how to give academic presentations and to provide productive comments and feedbacks. Every week, one student was scheduled to present an academic paper of his or her interest and another was scheduled to present the research progress of the selected thesis topic. Lab meetings encouraged this process of academic presentations and discussions that benefited not only the presenters but also the audiences. Second, I learned how to work on a thesis research project with my advisor. Every week, I also had to meet with him in order to report my progress toward my thesis and then had to go back and work on his suggestions and directions. I remember he once taught me that "I might be an expert in the field but not on the topic on which you are working. We need to learn together along the way until we reach the destination." This statement encouraged me to believe in my own research potentials and commence doing research since then. Third, I learned how to prepare a high-quality academic paper to get accepted for publication in academic conferences and journal publishers.

\vspace{0.2cm}
During the years of study at the Computer Engineering Department, I published two academic papers - one in an international conference's proceedings and the other in an ACM journal. When I prepared to submit an academic paper for the first time, I had to do three main tasks. First, I reviewed most prominent papers related to my topic and as I was reviewing, I learned the ideas of leading researchers in the field on the topic and how they presented them in the papers. However, I needed to come up with my own ideas, design my own solutions and compare my work with others'. Second, I needed to turn the ideas into the code implementation in a sensor network platform. Third, I had to explain and organize everything I had learned in an academic paper. According to my advisor, a high-quality paper should not only allow the readers to understand the overall picture of the work, but also enable them to implement it into the code themselves. Therefore, I explained the data structures, algorithms, and communication packets so clearly that one could use all this information for further experimentation. As a result, our paper titled ``Energy-Efficient Gradient Time Synchronization for Wireless Sensor Networks'', was accepted for publication. In the paper, we designed an extended version of gradient time synchronization protocols that was more time-accurate and energy-efficient, while maintaining a ``gradient'' property. With the gradient property, geographically adjacent nodes are able to maintain minimal synchronization errors.

\vspace{0.2cm}
In the other paper, all the co-authors had different tasks to finish, such as literature review, performance evaluation, and mathematical proofs. I was responsible for the introduction and related work parts. Our dedication and collaboration as well as the journal reviewers' valuable comments all played important roles in strengthening this piece of work. As a result, our paper titled ``Desynchronization with an artificial force field for wireless networks'' was accepted to publish in ACM SIGCOMM's \textit{Computer Communication Review}. The desynchronization problem is analogous to a resource allocation problem in which nodes cooperate to take turns accessing to the same resource. In this paper, we provide a prove of convexity of this problem. Additionally, we design a desynchronization protocol, inspired by electromagnetic force field, that performs in a distributed manner, better scales with network sizes and densities and produces less desynchronization errors. Even after graduation, my interest and ambition to do research never abates. In 2013, I had a change to work on a research project with Associate Professor Dr. Teerasit Kasetkasem of Kasetsart University. In this project, we used a signal processing technique to track a moving object in a field given binary sensor observation. In this paper, I was fully responsible for the manuscript preparation and partly for experimental simulation. Finally, the paper titled ``A Moving Object Tracking Algorithm Using Support Vector Machines in Binary Sensor Networks'', was finally accepted for publication, marking my third publication. 

\vspace{0.2cm}
I desire to advance my study to a PhD in the US because of the following three main reasons. First and most importantly, I want to be a professional researcher in computer science in the future, either in an academic institution or in a research laboratory and a doctoral degree is an important precursor to the research profession. Second, I agree with Matt Welsh, previously a professor of Computer Science at Harvard University, about a PhD study. He suggests that ``You get an intense exposure to every subfield of Computer Science, and have to become the leading world's expert in the area of your dissertation work.'' For example, during my PhD study, I will have an opportunity to get exposed to a variety of academic subjects and research projects in computer science, such as Artificial Intelligence, Computer Graphics, Robotics, Databases, Systems, Software Engineering, Computational Science, etc., all of which will considerably expand my intellectual horizons in computer science. Moreover, the PhD study will train me to be an expert in the field of my dissertation through the educational systems and processes, together with my assiduous and persevering efforts. Third, I am conscious that studying at a PhD level requires an academically vibrant environment which includes surroundings with brilliant students and faculty members, as well as accessible academic conferences and seminars. In my opinion, all of these are prevalent in the US educational systems and universities.

% the following paragraphs are specific to a particular program in a university. need to change for each submission to each program
\vspace{0.2cm}
I aspire to become a PhD student at \department, \university, a prestigious university in the US, because I am particularly interested in its teaching and research. A graduate course, \textit{Computer Communications}, taught by Professor Dr. Ramesh Govindan or Assistant Professor Dr. Ethan Katz-Bassett, requires students to study a variety of papers ranging from the classical papers regarding the design of the Internet to the more modern and visionary work pertinent to data center networks or software-defined networking. From my experience, simply reading those papers does not provide a tangible benefit for students.; it is the discussion and brainstorming between the students and the teacher that can lead to great ideas and innovations. Of course, great ideas alone do not suffice because computer scientists have to implement them to evaluate their performance and functionality. Therefore, in this course, students are required to do term projects. For example, in Fall 2013, students were asked to design a new transport protocol that reduced latency in data centers. Another graduate course, \textit{Software-Defined Networking}, taught by Assistant Professor Dr. Minlan Yu, follows the same philosophy by having students explore classical and contemporary papers and finish a term project. In conclusion, I am excited to be part of these courses which provide students with theoretical and practical learning experience.

\vspace{0.2cm}
My current research interest includes Internet research and software-defined networks. Therefore, I am interested in the following research projects of the \labfirst \space at \department, \uniabbre. First, ``Mapping the Expansion of Google's Serving Infrastructure'' is an interesting experimental Internet measurement research project. Today's large-scaled web providers, such as Google, take advantage of content distribution networks (CDNs) in order to reduce the latency perceived by the Internet's users by distributing web serving infrastructures to various locations around the world. This project aims to enumerate the mapping between clients and serving infrastructures and quantify how effective the mapping algorithms are. This kind of projects gives me the idea not only whether CDNs work but also how well they work. In my opinion, the measurement techniques of this paper can be extended to analyze other providers which may use different DNS techniques.

\vspace{0.2cm}
I am also interested in Software-Defined Network (SDN) research. In my opinion, SDN is the future of computer networks because it allows network administrators or programmers to control overall behavior of the network through its control plane while letting the data plane of network devices send and receive data. From my experience with enterprise network infrastructures with hundreds of network ports, it is laborious to adjust the behavior of networks each time the policy has changed because I have to configure each device individually. With SDN, the network's intelligence is centralized to a control device where all the configurations and control take place. For example, in the paper ``SIMPLE-fying Middlebox Policy Enforcement Using SDN'', Assistant Professor et al. use SDN to enforce policies regarding middleboxes such as firewalls, VPN gateways, proxies, etc., above the layer 2 and 3 of TCP/IP at which SDN is supposed to function. Applying SDN to the application of middleboxes provides network officers with more flexibility and control. I am confident that my research experience and professional background will give me an advantage to do SDN research projects.

\vspace{0.2cm}
My plan after graduation with a doctoral degree is that I will look for a research or post-doc position that is related to the field of my dissertation in order to continue to accumulate research knowledge and experience. Therefore, within five years after graduation, I will become a real expert in the field and plan to lead my own research laboratory. Research experience gained during the PhD study and accumulated after graduation will play an important role in attracting funds and research students into my lab.

\vspace{0.2cm}
I would like to express my appreciation to the admission committee of \university \space for taking my statement of purpose and other application materials into consideration. I hope that the committee will be convinced that my educational background, academic and professional experience, and research motivation and ambition are sufficient evidences to suggest that I will be an excellent student of the PhD program and a competent researcher in computer science in the future.

\vspace{1cm}
\raggedleft Kittipat Apicharttrisorn
\end{document}