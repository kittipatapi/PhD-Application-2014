\documentclass[a4paper,10pt]{report}
\hyphenation{mathe-matics under-graduate know-ledge}
\usepackage{fullpage}
\usepackage{fancyhdr}
\pagestyle{fancy}
\fancyhead[L]{\small Kittipat Apicharttrisorn}
\fancyhead[C]{\bfseries  \large Personal History Statement}
\fancyhead[R]{\tiny \degree \space in Computer Science \\ \university}
\fancyfoot[C]{\thepage}
\topmargin = 10pt
\headheight = 14pt
\headsep = 14pt
\footskip = 30pt
%\usepackage{helvetica}
\newcommand{\university}{University of California, Berkeley}
\newcommand{\department}{Electrical Engineering and Computer Sciences Department}
\newcommand{\uniabbre}{UC Berkeley}
\newcommand{\degree}{Doctor of Philosophy}
\newcommand{\labfirst}{N/A}
\newcommand{\labfirstabbre}{N/A}
\begin{document}
%Please describe how your personal background and experiences inform your decision to pursue a graduate degree. In this section, you may also include any relevant information on how you have overcome barriers to access higher education, evidence of how you have come to understand the barriers faced by others, evidence of your academic service to advance equitable access to higher education for women, racial minorities, and individuals from other groups that have been historically underrepresented in higher education, evidence of your research focusing on underserved populations or related issues of inequality, or evidence of your leadership among such groups. The Personal History Statement is required for all applicants. Please note that the Personal History Statement should not duplicate the Statement of Purpose.

\vspace{0.4cm}
This personal history statement introduces the obstacles during my Bachelor's degree study and my transformed attitudes that lead to the achievement of the Master's degree. Then, this statement explains my aspiration to become a competent researcher and professor in the future.

\vspace{0.2cm}
During my undergraduate study, in addition to a number of Electrical Engineering subjects, I studied a wide range of mathematical subjects including four Calculus courses, a course on Probability, and another on Linear Algebra and Complex Numbers, all of which are basic principles of Computer Science. Moreover, I passed two courses on computer programming, data structures and algorithms, which are the knowledge required for a successful computer scientist. However, during the first three years of the study, although I enjoyed learning the subjects, I was so shiftless and unmotivated that I did not pay much attention to my academic records. Not until the beginning of the forth year did I decide to boost my GPA as I was conscious that after that year I had to apply for a job and the currently low GPA would preclude me from competing with other students. This consciousness encouraged me to attend classes more frequently, pay more attention to the study materials, and better prepare for the examinations. As a result, my semester GPAs of the forth year were able to stay in a good standing until I graduated. Unfortunately, the total GPA was unable to increase much and remained at 2.49/4.00. 

\vspace{0.2cm}
Then I decided to continue my education to the Master's degree in Computer Science at the Department of Computer Engineering, Chulalongkorn University. I had the following three main reasons. First, as a computer systems engineer, studying computer science would give me an advantage in terms of vocational knowledge and development. Second, this thesis-based program would allow me to gain research experience in computer science, which would be crucial to my PhD study in the future. Third, in this program, I would have a chance to study a wide range of computer science subjects from Theory of Computation and Computer Algorithms to Computer Networks and Distributed Systems. With my attitude to have an excellent academic record, I worked hard on studying materials, undertaking term projects and making progress toward my thesis work. As a result, I was able to earn a very good GPA of 3.75/4.00 in the Master's degree.

\vspace{0.2cm}
My decision to pursue the Master's degree was correct because I gained a lot of valuable research experience there. At the department, I was a member of the Ubiquitous Network laboratory under the supervision of Associate Professor Dr. Chalermek Intanagonwiwat, who was also my advisor. At this lab, I learned at least three priceless lessons of research experience. First, every week, one student was scheduled to present an academic paper of his or her interest and another was scheduled to present the research progress of the selected thesis topic. Through this process of academic presentations and discussions, I learned how to prepare and give well-organized and intelligible academic presentations and how to give useful comments and feedbacks for others. Second, I learned how to work on a thesis research project with my advisor. Everyweek I had to meet with him in order to report my progress toward my thesis and then had to go back and work on his suggestions and directions. I remember he once taught me that "I might be an expert in the field but not on the topic on which you are working. We need to learn together along the way until we reach the destination." This statement encouraged me to believe in my own research potentials and commence doing research since then. Third, I learned how to prepare a high-quality academic paper to win acceptance from academic conferences and journal publishers.

\vspace{0.2cm}
As a result, I desire to be a generous and competent professor like my adviser. I believe that every undergraduate student is qualified to be a capable graduate student if he or she has been inspired to do research and to advance the study into the Master's or doctoral degree. Moreover, Thailand needs a novel, national research structure that attracts more funds from industrial or commercial sectors to support academic research institutions which undertake projects that solve the industrial or commercial problems in return. I aspire to fulfil a missing piece of such a structure by bridging aeronautical industry with computer science research.
\end{document}