\documentclass[a4paper,10pt]{report}
\hyphenation{mathe-matics under-graduate know-ledge}
\usepackage{fullpage}
\usepackage{fancyhdr}
\pagestyle{fancy}
\fancyhead[L]{\small Kittipat Apicharttrisorn}
\fancyhead[C]{\bfseries  \large Statement of Objectives}
\fancyhead[R]{\small \degree \space (EECS) \\ \uniabbre}
\fancyfoot[C]{\thepage}
\topmargin = 10pt
\headheight = 14pt
\headsep = 14pt
\footskip = 30pt
%\usepackage{helvetica}
\newcommand{\university}{Massachusetts Institute of Technology}
\newcommand{\department}{Electrical Engineering and Computer Science Department}
\newcommand{\uniabbre}{MIT}
\newcommand{\degree}{Doctor of Philosophy}
\newcommand{\labfirst}{N/A}
\newcommand{\labfirstabbre}{N/A}
\begin{document}
%Please explain why you are a good candidate for graduate school. You should describe why you wish to attend graduate school, what you would like to study, and any research experience you have. Describe one or more accomplishments you are particularly proud of that suggest that you will succeed in your chosen area of research.

\vspace{0.4cm}
This statement of objectives is intended for use with my application to the \degree \space graduate program at the \department, \university. This document starts by portraying my educational background, of both the Bachelor and Master's degrees. Then, it briefly states my professional experience during the employment as a systems engineer and explains my research experience during the Master's degree study. Then my interest in \uniabbre's teaching and research is elaborated and finally my future plans after Ph.D. graduation are described. After finishing reading this statement of purpose, the committee will learn why I am qualified to be an excellent student of the program, what motivates me to pursue the doctoral degree at \uniabbre \space and why it is so important for my future profession that I earn this degree.

\vspace{0.2cm}
During my undergraduate study, in addition to a number of Electrical Engineering subjects, I studied a wide range of mathematical subjects including four Calculus courses, a course on Probability, and another on Linear Algebra and Complex Numbers, all of which are basic principles of computer science. Moreover, I passed two courses on computer programming, data structures and algorithms, which are the crucial knowledge of a successful computer scientist. For the Master's degree study, I passed eight credited computer science graduate courses. I studied two theoretical courses, namely Theory of Computation and Computer Algorithms, five systems courses namely Information Systems Architecture, Distributed Systems, Advanced Topics on Computer Networks (Multimedia, Wireless and Adhoc Networks), Embedded Systems, and Database Management Systems, and one Artificial Intelligence course. Moreover, I passed two non-credit courses - namely Computer Security and Special Topics on Distributed Systems (Service Computing). In sum, I earn a solid foundation in computer science as a result of my undergraduate and graduate study. 

\vspace{0.2cm}
In addition, I gain valuable research experience during my Master's degree study and I would like to explain three principal research skills in this letter. First, I learn the critical reading skills. As an important part of research methodology in computer science, literature reviewing is an everyday activity of graduate students. Researchers study research papers not only to understand the overall concepts but also to critique them, find weak points and discover hidden assumptions. With this critical reading, I can find a research opportunity hidden in a research paper and can think of ``what to do next'' instead of just ``this work is interesting''. Second, I learn how to give an intelligible academic presentation. At the UbiNet lab under the supervision of Assistant Professor Dr. Chalermek Intanagonwiwat, each lab member took turns giving one progress presentation reporting the progress toward the thesis work and one paper presentation illustrating the ideas and results of a research paper of interest. Through this regular lab activity, I learned to select an interesting paper published in a well-known conference or journal publisher, to extract outstanding points in the paper and to present them in a way that attracted attention from the audience. Third, after completing a certain amount of literature review and algorithm design and implementation, I need to publish a paper in order to organize my ideas into a standard format, to distribute my work for other researchers to study and to welcome feedbacks and comments from reviewers. According to my advisor, a high-quality paper in computer science should not only allow the readers to understand the overall picture of the work, but also enable them to implement it into the code themselves. Therefore, I learn to explain the data structures, algorithms, and communication packets so clearly that one could use all this information for further experimentation.

\vspace{0.2cm}
Up to now, I have published three academic publications, two of which are in international conferences' proceedings and the other is in an ACM journal.  First, ``Energy-Efficient Gradient Time Synchronization for Wireless Sensor Networks'' was published in the proceedings of the Second International Conference on Computational Intelligence, Communication Systems and Networks or CICSyN. In the paper, we designed an extended version of gradient time synchronization protocols that was more time-accurate and energy-efficient, while maintaining a ``gradient'' property. With the gradient property, geographically adjacent nodes are able to maintain minimal synchronization errors. Second, ``Desynchronization with an artificial force field for wireless networks'' was published in ACM SIGCOMM's \textit{Computer Communication Review}. The desynchronization problem was analogous to a resource allocation problem in which nodes cooperated to take turns accessing to the same resource. In this paper, we provided a prove of convexity of this problem. Additionally, we designed a desynchronization protocol, inspired by electromagnetic force field, that performed in a distributed manner, better scaled with network sizes and densities and produced less desynchronization errors. The first two papers were my work under the supervision of Assistant Professor Dr. Chalermek Intanagonwiwat. Third, in 2013, I had a change to work on a research project with Associate Professor Dr. Teerasit Kasetkasem of Kasetsart University. In this project, we used a signal processing technique to track a moving object in a field given binary sensor observations. In this paper, I was fully responsible for the manuscript preparation and partly for experimental simulation. Finally, the paper titled ``A Moving Object Tracking Algorithm Using Support Vector Machines in Binary Sensor Networks'', was finally published in the proceedings of The 13th International Symposium on Communications and Information Technologies.

\vspace{0.2cm}
I also have seven-year professional experience working at Aeronautical Radio of Thailand or Aerothai, a state enterprise under the Ministry of Transport, Thailand. One of Aerothai's principal missions is to provide safe and efficient air navigation services or air traffic control within Thailand's airspace. Specifically, the department of air traffic data systems engineering is responsible for the provision and administration of data systems that support air traffic controllers' operations safely and efficiently. At the department, my colleagues and I design, configure, and implement those systems by taking advantage of enterprise-graded information system products, mostly of the USA, such as HP and Dell servers, Oracle and Microsoft databases, Cisco network equipment, and VMWare's virtualization technology, etc. One of the interesting aeronautical applications that runs on these infrastructures is the flight scheduling service, named Bay of Bengal Cooperative Air Traffic Flow Management System or BOBCAT. BOBCAT manages the air traffic over the Bay of Bengal, which has the security constraints. Approximately 60 flights per day request to fly through this narrow airspace; therefore, International Civil Aviation Organization or ICAO demands that the airspace be managed by Aerothai, after the architectural and algorithmic competition with other organizations. Nowadays, BOBCAT smoothly serves tens of airline customers requesting air space slots over the area every day thanks to Aerothai's effective software systems and responsive operational procedures.  Therefore, I have witnessed how these innovative products help enhance reliability and efficiency of air traffic data systems. This hand-on experience has provided me with practical aspects of enterprise information systems with the safety-critical applications, and motivates me to study more deeply and broadly in computer science, a core foundation of computer-related products and services. 

\vspace{0.2cm}
I determine to advance my study to a PhD in the US because of the following three main reasons. First and most importantly, I want to be a professional researcher in computer science in the future, either in an academic institution or in a research laboratory and a doctoral degree is an important precursor to the research profession. Second, I agree with Matt Welsh, previously a professor of Computer Science at Harvard University, about a PhD study. He suggests that ``You get an intense exposure to every subfield of Computer Science, and have to become the leading world's expert in the area of your dissertation work.'' For example, during my PhD study, I will have an opportunity to get exposed to a variety of academic subjects and research projects in computer science, such as Artificial Intelligence, Computer Graphics, Robotics, Databases, Systems, Software Engineering, and Computational Science, etc., all of which will considerably expand my intellectual horizons in computer science. Moreover, the PhD study will train me to be an expert in the field of my dissertation through the educational systems and processes, together with my assiduous and persevering efforts. Third, I am conscious that studying at a PhD level requires an academically vibrant environment which includes surroundings with brilliant students and faculty members, as well as accessible academic conferences and seminars. In my opinion, all of these are prevalent in the US educational systems and universities.

% the following paragraphs are specific to a particular program in a university. need to change for each submission to each program
\vspace{0.2cm}
I aspire to become a PhD student at \department, \university, a prestigious university in the US, because I am particularly interested in its teaching and research. To begin with the teaching, the graduate course ``Computer Networks'' is of my particular interest. In the course, students get a chance to study academic papers ranging from classical work contributing to the earlier success of the Internet to contemporary work of today's ``hot'' topics in computer network research, such as datacenter networks or software-defined networks. Each week, students are required to study a few papers in order to gain prerequisite knowledge before participating in the class discussion. From my research experience, this discussion on academic papers can enrich the students' skills to conduct their own research such as critical reading and critical thinking. Another graduate subject that interests me is ``Data Communication Networks''. In this course, I will learn the basic fundamentals of various network protocols involving different network layers. Most interestingly, the queuing theory that plays an important role in today's network predictability is also covered. In addition to network-related subjects, I am also enthusiastic about other systems courses such as ``Principles of Computer Systems'', or theoretical subjects such as ``Network Optimization'' or artificial intelligence courses such as ``Machine Learning and Neural Networks''. In particular, I am working on an independent research project with Associate Professor Dr. Teerasit Kasetkasem of Kasetsart University. In this project, we use a machine learning technique called support vector machine to track a moving object in a binary sensor network field. Therefore, I realize how knowledge from one area can be used to solve a problem in another area. In sum, I am academically interesting in EECS's Area II with the concentration on \textit{Systems}.

\vspace{0.2cm}
The following are \uniabbre's faculty whose research projects interest and excite me. First, Professor Dina Katabi's famous ``See Wi-Vi: See Through Walls with Wi-Fi Signals'' project is truly innovative. She and the team use the analysis of ubiquitous Wi-Fi signal to track a moving object even when it is behind the wall. They use specialized tools and algorithms to detect a change in signal patterns as an object moves. The applications of Wi-Vi include intrusion detection and police or rescue operations. However, in my opinion, this project has much work left to be further explored. For example, how does Wi-Vi handle multiple objects moving at the same time? How does it track an object in the presence of a wireless signal jammer? Moreover, her best paper at SIGCOMM 2011 titled ``They Can Hear Your Heartbeats: Non-Invasive Security for Implanted Medical Devices'' was once presented in a UniNet lab meeting at Chulalongkorn University by a lab member. This paper describes the fatal vulnerability of implantable medical devices to wireless attacks and proposes a novel solution to the problem. These two projects are a good example of Professor Dina Katabi's insights into wireless signal and communications. 

\vspace{0.2cm}
Second, Professor Hari Balakrishnan's ``TCP ex Machina: Computer-Generated Congestion Control'' project proposes a new method to solve the network congestion control problem of TCP. Instead of traditional manual, default TCP configurations at endpoints, a program called Remy generates congestion control codes for endpoints by considering both network knowledge and assumptions and endpoints' objectives such as low latency or high throughput. In the experiments, Remy outperforms traditional TCP congestion control mechanisms and even those that change intermediate nodes and break ``end-to-end arguments''. In my opinion, this project can be further developed with the software-defined networks (SDN). Remy can use not only a local view of the network from the endpoints but also a global or sub-global view from the SDN's control planes in order to generate congestion control configuration that reflects the current and global behavior of network congestion. 

\vspace{0.2cm}
Third, Professor Nancy Lynch's ``Reliable communication, unreliable networks'' project recently makes a contribution in wireless networks from both theoretical and practical perspectives. It proves cost bounds of radio broadcasts in unreliable networks in the presence of adversary nodes. It also proposes several adversary models including a practical one called \textit{oblivious}, in which an adversary node only observes the behavior of the network without knowledge of in-network algorithmic executions. These bounds and models can help network designers to better understand the behavior and effects of adversary nodes in the wireless network. In my opinion, network researchers working on this kind of projects have to gain a solid background in graph theory and a practical understanding in wireless, ad-hoc networks. In sum, it will be a great opportunity for me to work with these distinguished professors who have insights into computer science research.

\vspace{0.2cm}
My plan after graduation with a doctoral degree is that I will look for a research or post-doc position that is related to the field of my dissertation in order to continue to accumulate research knowledge and experience. Therefore, within five years after graduation, I will become a real expert in the field and plan to lead my own research laboratory. Research experience gained during the PhD study and accumulated after graduation will play an important role in attracting funds and students into my lab.

\vspace{0.2cm}
I would like to express my appreciation to the graduate admission committee of \university \space for taking my statement of purpose along with other application materials into consideration. I hope that the committee will be convinced that my educational background, academic and professional experience, and research ambition and motivation are the evidences sufficient to suggest that I will be an excellent student of the PhD program and a competent researcher in computer science.

\end{document}